\section{Přednáška 4 -- Optimalizační problémy}

Optimalizační problém patří do NPO, pokud:

\begin{itemize}
    \item výstup lze zapsat,
    \item omezující podmínky lze vyhodnotit v polynomiálním čase,
    \item optimalizační kritérium lze vyhodnotit v polynomiálním čase.
\end{itemize}

\subsubsection*{Třída PO}

Optimalizační problém patří do PO, jestliže patří do NPO a zároveň existuje Turingův stroj řešící každou instanci v polynomiálním čase.
Příkladem může být problém nejkratší cesty v grafu.

\subsubsection*{Pseudopolynomiální algoritmy}

Jedná se o algoritmus, jehož počet kroků závisí polynomiálně na velikosti instance, ale závisí také na parametru, který s velikostí instance nesouvisí.

\subsubsection*{Aproximativní algoritmus, třída APX}

Aproximativní algoritmus pro problém $\Pi$ je $\epsilon$-aproximativní, jestliže každou instanci vyřeší v polynomiálním čase s relativní chybou $\epsilon$.
Číslo $\epsilon$ se nazývá aproximační práh.

Optimalizační problém patří do třídy APX, jestliže je $R$-aproximativní pro konečné $R$.

Platí, že $\epsilon = 1 - \frac{1}{R}$.

\subsubsection{PTAS -- Polynomial Time Approximation Scheme}

Algoritmus APR který, pro každé $1 > \epsilon > 0$ vyřeší každou instanci problému s relativní chybou nejvýše $\epsilon$ v polynomiálním čase nazýváme polynomiální aproximační schéma.

\subsubsection{FPTAS -- Fully Polynomial Time Approximation Scheme}

Polynomiální aproximační schéma APR, jehož čas výpočtu závisí polynomiálně na $1/\epsilon$ se nazývá plně polynomiální aproximační schéma.

\subsubsection{Randomizované algoritmy}

Založeny na náhodné volbě, jejich výsledek je vyjádřen statisticky:

\begin{itemize}
    \item dosažený výsledek je náhodná proměnná, čas běhu je pevný pro dannou instanci,
    \item čas běhu je náhodná proměnná, výsledek vždy správný.
\end{itemize}

Výhody randomizovaných algoritmů:

\begin{itemize}
    \item očekávaná kvalita výsledku může být lepší než zaručená kvalita aproximativních algoritmů,
    \item nezávislým opakováním se dá kvalita výsledku zlepšovat.
\end{itemize}
