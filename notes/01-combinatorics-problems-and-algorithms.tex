\section{Přednáška 1 -- Kombinatorické problémy a algoritmy}

\subsection{Kombinatorická matematika, problémy}

\begin{itemize}
    \item Konečné a diskrétní problémy -> konečný počet proměnných s konečným počtem hodnot pro každou z nich
    \item Hledá řešení pomocí zkoušení všech možných hodnot
    \item Poskytuje výsledek v konečném, ale vysokém čase
\end{itemize}

\subsubsection{Názvosloví}

\begin{description}
    \item[Vstupní proměnné] Proměnné charakterizující problém (počet prvků v batohu, nosnost, ohodnocení a hmotnost prvků)
    \item[Výstupní proměnné] Proměnné očekávané v rámci řešení problému (seznam věcí v batohu $x_i \in {0,1}$)
    \item[Instance problému] Ohodnocení vstupních proměnných
    \item[Konfigurační proměnné] Proměnné nastavované při hledání řešení, zkoumá se zda splňují omezení (seznam věcí v batohu $x_i \in {0,1}$)
    \item[Konfigurace] Ohodnocení konfiguračních proměnných ($x_1 := 1, x_2:= 0, \ldots, x_n = 0$)
    \item[Omezení] Podmínky, které musí konfigurace splňovat aby byla řešením (součet vah věcí v batohu nesmí překročit nosnost)
    \item[Optimalizační kritérium] Podmínky kladené na optimální řešení - určuje kdy je řešení optimální (cena věcí musí být v součtu maximální)
    \item[Řešení, suboptimální řešení, optimální řešení] Instance která splňuje omezení, instance která splňuje omezení a má vyhovující hodnotu optimalizačního kritéria, konfigurace splňující omezení mající nejlepší hodnotu optimalizačního kritéria
\end{description}

\subsubsection{Verze kombinatorických problémů}

Mějme instanci $I$, konfiguraci $Y$ a omezení $R(I, Y)$ řikající, zda $Y$ je řešením.

\begin{description}
    \item[Rozhodovací problém] \
    \begin{itemize}
        \item Existuje $Y$ takové, že $R(I, Y)$ je řešením?
        \item Platí pro všechna $Y$, že $R(I, Y)$ je řešením?
    \end{itemize}

    \item[Konstruktivní problém] \
    \begin{itemize}
        \item Sestrojit $Y$, pro které $R(I, Y)$
    \end{itemize}

    \item[Enumerační problém] \
    \begin{itemize}
        \item Najít všechna $Y$, že $R(I, Y)$
    \end{itemize}
\end{description}

Tyto verze mají stejné vstupní a konfigurační proměnné a omezení $R$.
Liší se pouze výstupními proměnnými.
Díky tomu je lze převádět a rozpoznat.

\subsection{Optimalizační problémy}

Mějme instanci $I$, konfiguraci $Y$ a omezení $R(I, Y)$ řikající, zda $Y$ je řešením, $C(Y)$ je cenová funkce.

\begin{description}
    \item[Optimalizační rozhodovací problém] \
    \begin{itemize}
        \item Existuje $Y$ takové, že $R(I, Y)$ a $C(y)$ je alespoň tak dobré jako konstanta $Q$?
    \end{itemize}

    \item[Optimalizační konstruktivní problém] \
    \begin{itemize}
        \item Sestrojit $Y$, pro které $R(I, Y)$ a $C(Y)$ je nejlepší možné.
    \end{itemize}

    \item[Optimalizační enumerační problém] \
    \begin{itemize}
        \item Najít všechna $Y$, že $R(I, Y)$ a $C(Y)$ je nejlepší možné.
    \end{itemize}

    \item[Optimalizační evaluační problém] \
    \begin{itemize}
        \item Najít nejlepší možné $C(Y)$ takové, že $R(I, Y)$.
    \end{itemize}
\end{description}

\subsection{Problém splnitelnosti booleovské formule - SAT}

Pro $n$ proměnných booleovské formule v konjuktivní normální formě (součin součtů) určit, zda je splnitelná.

\subsubsection{Verze SAT problému}

\begin{description}
    \item[Rozhodovací problém] \
    \begin{itemize}
        \item Existuje ohodnocení takové, že je formule splnitelná?
    \end{itemize}

    \item[Konstruktivní problém] \
    \begin{itemize}
        \item Sestrojit ohodnocení takové, že je formule splnitelná.
    \end{itemize}

    \item[Optimalizační Enumerační problém] \
    \begin{itemize}
        \item Sestrojit všechna ohodnocení taková, že je formule splnitelná.
    \end{itemize}
\end{description}

Pro optimalizační verze se následně určuje, zda existuje ohodnocení, mající nejvýše $Q$ jedniček resp. co nejméně jedniček.

\subsection{Složitost problému}

\begin{itemize}
    \item Asymptotická horní mez
    \begin{itemize}
        \item $f(n) = O(g(n)) \Leftrightarrow \exists c > 0, n_0 \in N: \forall n > n_0: f(n) \leq c \cdot g(n)$
    \end{itemize}

    \item Asymptotická horní mez
    \begin{itemize}
        \item $f(n) = \Omega(g(n)) \Leftrightarrow \exists c > 0, n_0 \in N: \forall n > n_0: f(n) \geq c \cdot g(n)$
    \end{itemize}

    \item Asymptotický odhad
    \begin{itemize}
        \item $f(n) = \Theta(g(n)) \Leftrightarrow f(n) = O(g(n)) \land f(n) = \Omega(g(n))$
    \end{itemize}
\end{itemize}
