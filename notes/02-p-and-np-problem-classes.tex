\section{Přednáška 2 -- Třídy P a NP}

\begin{description}
    \item[P problémy] Únosné problémy, řešitelné v polynomiálním čase na \textbf{deterministickém} Turingově stroji
    \item[NP problémy] Únosně velký prostor konfigurací, řešitelné v polynomiálním čase na \textbf{nedeterministickém} Turingově stroji, zkratka pro \texttt{Non-Deterministic Polynomial}
\end{description}

\subsection{Třída problémů P}

\begin{itemize}
    \item Program řeší na deterministickém Turingově stroji rozhodovací problém, pokud se jestli že se výpočet zastaví po konečném počtu kroků pro každou instanci tohoto problému.
    \item Program řeší na DTS rozhodovací problém v čase $t$, pokud se pro každou instanci zastaví po $t$ krocích.
    \item Program řeší na DTS rozhodovací problém s pamětí $m$, jestliže pro každou instanci využije nejvíce $m$ políček pásky.
\end{itemize}

\subsubsection*{PSPACE}
Problém lze vyřešit v paměti $O(n^k)$, kde $n$ je velikost instance a $k$ konečné číslo.
Označuje třídu problémů řešitelných v polynomiální paměťové složitosti.

\subsubsection*{EXTIME}
Problém lze vyřešit v čase $O(2^{P(n)})$, kde $P(n)$ je polynom ve velikosti instance $n$.
Označuje třídu problémů řešitelných v exponenciální časové složitosti.

\subsection{Třída problémů NP}


Rozhodovací problém spadá do třídy NP:
\begin{itemize}
    \item jestliže pro něj ezistuje NTS, který pro každou instanci mající výstup \textit{ano} řeší v čase $O(n^k)$,
    \item jestliže pro každou instanci $I \in \Pi_{ano}$ existuje konfigurace $Y$ taková, že kontrola zda $Y$ je řešením patří do P
\end{itemize}

\begin{description}
    \item[Cetifikát] Konfigurace NTS, jejíž ověření že je výsledkem patří do P.
    Tato konfigurace musí existovat pro \textbf{každou} instanci $I \in \Pi_{ano}$.
\end{description}

\subsubsection*{Nedeterministický Turingův stroj}

Lze si představit jako klasický Turingův stroj, který v momentě kdy se má rozhodnou se naklonuje a každý z klonů se vydá jednou cestou.
Jedna z kopií najde únosně dlouhou cestu prostorem konfigurací -- tj. najde řešení.

Jiná verianta je, že tento Turingův stroj v momentě kdy se musí rozhodnout vždy vybere náhodně správnou možnost.

\begin{itemize}
    \item Problém je řešitelný NTS v čase $t$, jestliže se výpočet zastaví po $t$ krocích pro každou instaci $I$, mající výstup ano.
\end{itemize}

Jestliže je problém řešitelný na NTS v čase $T(n)$, pak na $DTS$ je řešitelný v $2^{O(T(n))}$.

\subsubsection*{Hamiltonovská kružnice}

Hledání Hamiltonovské kružnice je problém nalezení kružnice v grafu, která prochází všemi vrcholy právě jednou.

\begin{itemize}
    \item Konfigurace - Podgraf $G = (V', E')$ původního grafu
    \item Krok algoritmu - V každém kroku se přidá:
    
    \begin{enumerate}
        \item hrana $e = (u, v)$ taková, že nenáleží $E'$ a zároveň $u \ V', v \notin V', \textrm{deg}(v) = 1$
        \item uzel $v$,
        \item pokud to není možné, \uv{kopie} NTS končí,
        \item pokud přidaná hrana vytvoří v $G'$ kružnici kratší než $|V|$, \uv{kopie} končí,
        \item pokud přidaná hrana vytvoří v $G'$ kružnici délky $|V|$, algoritmus vydá výstup \uv{ano}
    \end{enumerate}
\end{itemize}

Pro nedeterministický algoritmus platí HC platí, že pokud graf obsahuje kružnici pak je nalezena po $|V|$ krocích.
Navíc také platí, že pokud nalezena není, pak neexistuje (neplatí u algoritmů obecně).

Certifikát HC lze ověřit v polynomiálním čase: $G'$ obsahuje $|V|$ uzlů, $E' \subset E$ a žádný uzel se ve $V'$ nevyskytuje dvakrát.

\subsection{Třída problémů co-NP}

Jedná se o množinu problémů komplementárních (opačných) k problémům NP.
Pokud se u NP problémů ptáme na odpověď ano, u co-NP se ptáme na odpověď \uv{ne} (HC: je graf \textit{prost} Hamiltonovských kružnic, SAT: je formule \textit{nesplnitelná}).
Doplňky omezujících pravidel se vytváří Demorganovým pravidlem.

\begin{itemize}
    \item NP: $\exists Y, R(I, Y)$
        \begin{itemize}
            \item \textbf{Existuje} Hamiltonovská kružnice?
            \item \textbf{Existuje} ohodnocení, pro které je formule splnitelná?
        \end{itemize}

    \item co-NP: $\forall Y, R'(I, Y$
    \begin{itemize}
        \item Prochází \textbf{všechny} kružnice méně než $|V|$ uzly?
        \item Platí pro \textbf{každé} ohodnocení $F(X) = 0$?
    \end{itemize}
\end{itemize}

Existují problémy, které jsou ve skupinách NP i co-NP.
Tím je například faktorizace čísla (dáno $N$, existuje prvočinitel jehož poslední číslo je 7? ANO: rozklad na prvočinitele, NE: rozklad na prvočinitele).
