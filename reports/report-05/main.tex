\documentclass[czech]{article}

\usepackage{natbib}
\usepackage[utf8]{inputenc}
\usepackage[IL2]{fontenc}
\usepackage{a4wide}
\usepackage[czech]{babel}
\usepackage{amsmath}
\usepackage{amsfonts}
\usepackage{amstext}
\usepackage{hyperref}
\usepackage{graphicx}
\usepackage{float}

\title{BI-PAA - Problém vážené splnitelnosti booleovské formule, úloha 5}
\author{Josef Doležal}

% Illustration
\newcommand{\image}[3]{
    \begin{figure}[H]
        \centering
        \includegraphics[width=0.8\textwidth]{assets/#3}
        \caption{#2}
        \label{fig:#1}
    \end{figure}
}

\begin{document}

\maketitle

\newpage

\section{Úvod}
Problémem nalezení řešení vážené splnitelnosti boolevské formule se nazývá úloha, ve které je za úkol nalézt ohodnocení proměnných $x_1, \dots, x_n$ formule v konjuktivním normální tvaru takové, pro které je formule splnitelná a součet vah proměnných je maximální.

Vstup tedy je množnina proměnných, kde každá má kladné ohodnocení.
Dalším vstupem je množina klauzulí určující celkový tvar formule.

Výstupem je číslo určující součet vah literálů, které jsou ohodnoceny jedničkou.
Ze všech možných kombinací je správné řešení takové, kde součet ohodnocení je maximální.

Tento problém je označován jako SAT (\textit{the SATisfiability problem}) a jeho speciálním případem je 3SAT, tedy formule kde každá klauzule má právě tři proměnné.
Dle Cookovi věty je SAT \textit{NP-úplný} problém \citep{sat-npc}.

\section{Zadání úlohy}

Je dána booleovská formule $F$ proměnnných $X=(x_1, x_2, \dots, x_n)$ v konjunktivní normální formě (tj. součin součtů).
Dále jsou dány celočíselné kladné váhy $W=(w_1, w_2, \dots, w_n)$.
Najděte ohodnocení $Y=(y_1, y_2, \dots, y_n)$ proměnných $x_1$, $x_2$, \dots, $x_n$ tak, aby $F(Y)=1$ a součet vah proměnných, které jsou ohodnoceny jedničkou, byl maximální.

Je přípustné se omezit na formule, v nichž má každá klauzule právě 3 literály (problém 3 SAT).
Takto omezený problém je stejně těžký, ale možná se lépe programuje a lépe se posuzuje obtížnost instance.

\section{Rámcový popis řešení}

Z definice plyne, že pro tento problém neexistuje efektivní algoritmus, který by v polynomiálním čase vrátil správné řešení nebo rozhodl o neřešitelnosti úlohy.
Z tohoto důvodu je k řešení využita pokročilá heauristika \textit{Simulované ochlazování}.

Řešení úlohy je pomocí této optimalizační metody je inspirováno ve světě fyziky, konkrétně založené na metodě simulace žíhání oceli.
Tato metoda si klade za cíl neuváznout při hledání řešení v lokálním maximu.
Toho se tato metoda snaží dosáhnout tak, že v závislosti na aktuálním nastavení a náhodném rozhodnutí dočasně přijme i takové řešení, které je horší než aktuálně nalezené.
Tím se liší od běžných heuristik, které běžně přijímají jen řešení lepší než aktuálně nalezené a tím jsou náchyl k uváznutí v lokálních extrémech.

\subsection{Popis algoritmu}

Tento algoritmus se skládá z hlavních částí.
První částí je aktuální teplota -- ta určuje jak moc je algoritmus v dannou chvíli \uv{ochotný} přijmout horší řešení.
Čím vyšší teplota je, tím spíše algoritmus neoptimální řešení přijme.
Po dobu běhu algoritmu teplota stále klesá a tím klesá i \uv{ochota} přijímat horší řešení.

Druhou částí je volba výchozího řešení -- náhodné konfigurace.
Tato konfigurace je volena zcela náhodně a nemusí nutně být řešením problému, je ale nutné k hledání sousedních řešení.

Třetí částí je \textit{equilibrium}, které určuje jak dlouho algoritmus setrvá v hledání řešení než dojde ke snížení teploty.

V dalším kroku dochází ke hledání \textit{sousedního} řešení.
Sousedním řešením se rozumí takové řešení, které má ve stavovém prostoru hranu vedoucí k aktuálnímu řešení -- tedy změna ohodnocení jedné konkrétní proměnné.

Nová konfigurace je přijata jako aktuální řešení, pokud řešením a zároveň je lepší než aktuálně nalezené řešení.
Kvalita řešení je určena hodnotící funkcí, která je dána jako součet vah proměnných ohodnocených jedničkou.

V případě kdy je nové řešení horší než aktuální, počítá se pravděpodobnost přijetí na základě aktuální teploty.
Pravděpodobnost je vypočítána podle vzorce $a = e^\frac{c_{old} - c_{new}}{t}$, kde $c_{old}$ je aktuální řešení, $c_{new}$ je nově nalezené řešení a $t$ je aktuální teplota.
Tato rovnost je inspirována ve fyzice, kde popisuje pohyb částic v oceli při určité teplotě.
Výsledek rovnosti je následně porovnán s náhodným čísle, je-li $a$ větší než náhodné číslo z intervalu $[0, 1]$, pak je konfigurace přijata.

Po setrvání na jedné teplotě po předem definovanou dobu je snížena teplota a algoritmus prochází znovu od kroku 2.
Algoritmus se zastaví po dosažení minimální definované teploty.

Algoritmus se tedy skládá ze dvou vnořených smyček.
Vnější reprezentuje aktuální teplotu.
Vnitřní smyčka pro každou teplotu vybírá náhodné sousední řešení a určuje, jak dlouho toto řešení hledat.
Počet iterací vnější smyčky je dán ochlazováním aktuální teploty.

\subsection{Popis implementace}

Pro implementaci jsem zvolil programovací jazyk Swift, který nabízí efektivní práci nejen z IDE, ale i příkazové řádky.
Vstupní soubory vyžadují \textit{DIMACS} formát se zadanými váhami proměnných.

Algoritmus pracuje s datovou strukturou instance \texttt{SATInstance}, která obsahuje množinu klauzulí a množinu literálů.
Ty jsou reprezentovány rozhraním \texttt{Literal}, které implementuje proměnná \texttt{Formula} (zde jsem nezvolil adekvátní název, nicméně refaktorování Swiftu je poněkud nešťastné a tak jsem název ponechal) a negace proměnné \texttt{Not}.
Negace si pouze drží referenci na literál který neguje, při změně hodnoty literálu se tedy změní ve všech klauzulích a patřičným způsobem se aplikují i negace.

Struktura instance nabízí v rámci svého rozhraní metodu \texttt{solution} resp. \texttt{configurationValue}, vracející aktuální řešení resp. konfiguraci.
Nové řešení algoritmus hledá prohozením hodnoty náhodného literálu a následným vypočítáním nové ceny a splnitelnosti.
Splnitelnost formule je hledána průchodem všech klauzulí.
Tuto část algoritmu jsem implementoval ve funkcionálním stylu (s využitím \texttt{map}, \texttt{filter} a \texttt{reduce}), což se při měření časové složitosti ukázalo jako velmi nevhodná volba.

Vzhledem k využití množin jako kolekcí pro uložení literálů umí program zpracovat klauzule s libovolným počtem literálů či jejich negací, neni tedy omezen 3SAT.
Všechna měření jsem ale prováděl právě na instancích 3SAT problému.

Správnost algoritmu jsem ověřoval u vybraných instancí algoritmem hrubé síly.
Vzhledem k časové náročnosti jsem ale neprováděl rozsáhlé testování odchylek.

Pro zvýšení přesnosti algoritmus zkusí v prvním kroku \uv{uhádnout} konfiguraci, která je zároveň řešením.
V měřeních jsem nezaznamenal významný dopad na časovou složitost, nicméně přesnost výsledného řešení se zvýšila.

Jako další optimalizaci výsledného řešení jsem implementoval paměť nejlepšího nalezeného řešení, které je vráceno v případě, že algoritmus se vlivem přijetí horšího výsledku přesunul do lokálního maxima.

\subsection*{Měnitelné parametry}

Z výše uvedeného popisu algoritmu vyplývá, že měnitelnými parametry algoritmu jsou:

\begin{itemize}
    \item počáteční teplota,
    \item počet hledání sousedního řešení pro aktuální teplotu,
    \item rychlost ochlazování aktuální teploty.
\end{itemize}

Pro každý z nich v práci zkoumám časovou náročnost programu pro různý počet literálů.

\section{Výsledky měření}

Měření znázorňují průměrné hodnoty pro $20$ instancí problému.
Časová složitost byla měřena jako procesorový čas, který byl potřebný přímo k výpočtu.
Každý graf sleduje změnu parametru pro více velikostí vstupu (konkrétně $27, 30, 32, 35, 37$ a $40$).

\subsection{Generování vstupů}

Program obsahuje rozhraní pro ovládání z příkazové řádky (nápověda se zobrazí po spuštění s přepínačem \texttt{--help}).
Jako vstup je možné zadat parametry pro konfiguraci algoritmu simulovaného ochlazování a seznam vstupních souborů.
Výstupem je číslo udávající průměrný čas potřebný pro výpočet jednoho ze zadaných vstupů.

Vstupní soubory jsem generoval programem dostupným z \cite{fw-gen}.
Generování je automatizováno naskriptovanými scénáři, které je možné spustit příkazem \texttt{make}:

\begin{itemize}
    \item \texttt{fixtures} -- generuje adresářovou strukturu a vstupy pro měření,
    \item \texttt{measure} -- spustí měření časové složitosti programu,
    \item \texttt{graphs} -- vygeneruje z naměřených hodnot grafy programem \texttt{gnuplot}.
\end{itemize}

\subsection{Počáteční teplota}

Počáteční teplotu algoritmu jsem volil podle vzorce $n*k$, kde $n$ je počet literálů a $k$ a parametr měněný v rámci měření.
Protože je měření prováděno pro více $n$, zvolil jsem jako referenční $n$ hodnotu $25$.

\subsubsection*{Konfigurace}

\begin{tabular}{ | l | l | }
    \hline
    Parametr & Hodnota \\ \hline \hline
    Počáteční teplota & $[225, 350]$, $k \in 9\dots14$ \\
    Iterace vnitřní smyčky & $400$ \\
    Rychlost ochlazování & $0.95 \cdot \textrm{aktuální teplota}$ \\ \hline
\end{tabular}

\subsubsection*{Rychlost běhu}

\image{initial-temperature}{Časová závislost výpočtu na počáteční teplotě}{initial-temperature.png}

\subsubsection*{Komentář}

Z grafu \ref{fig:initial-temperature} lze vyčíst, že až na jednu výjimku je růst převážně lineární.
Dle očekávání mají časovou složitost vyšší ty instance, kde počáteční teplota byla vyšší (dáno vyšším počtem vnějších cyklů).
Za zmínku stojí rozpětí hodnot.
Rozdíl nejpomalejšího a nejrychlejší běhu je pouse $30$ ms, což je vzhledem na počet průchodů jen malá změna.

\subsection{Iterace vnitřní smyčky}

\subsubsection*{Konfigurace}

\begin{tabular}{ | l | l | }
    \hline
    Parametr & Hodnota \\ \hline \hline
    Počáteční teplota & $400$ \\
    Iterace vnitřní smyčky & $[100\dots500]$ \\
    Rychlost ochlazování & $0.95 \cdot \textrm{aktuální teplota}$ \\ \hline
\end{tabular}

\subsubsection*{Rychlost běhu}

\image{equilibrium}{Časová závislost výpočtu na průběhu vnitřní smyčky}{equilibrium.png}

\subsubsection*{Komentář}

Obrázek \ref{fig:equilibrium} ukazuje podle očekávání, že časová složitost roste s počtem vnitřních cyklů.
Je ale naopak také patrné, že celkový běh není pro konkrétní velikost vnitřního cyklu nijak výrazně závislý na počtu prvků.

\subsection{Rychlost ochlazování}

\subsubsection*{Konfigurace}

\begin{tabular}{ | l | l | }
    \hline
    Parametr & Hodnota \\ \hline \hline
    Počáteční teplota & $400$ \\
    Iterace vnitřní smyčky & $400$ \\
    Rychlost ochlazování & $0.8\dots0.99 \cdot \textrm{aktuální teplota}$ \\ \hline
\end{tabular}

\subsubsection*{Rychlost běhu}

\image{annealing-factor}{Časová závislost výpočtu na rychlosti ochlazování}{annealing-factor.png}

\subsubsection*{Komentář}

Z grafů \ref{fig:annealing-factor} lze vyčíst, že přestože rychlým ochlazováním rapidně klesá časová složitost, relativní naopak roste.
Z naměřených výsledků se jako optimální faktor snížení teploty jeví násobek $0.9$.
Zajímavým pozorováním je růst časové složitosti pro instance o $37$ prvcích, kde se pro všechna měření zvedl čas téměř sedminásobně.

\section{Závěr}

Naměřená data pro časovou náročnost odpovídají předpokladům, že se zvyšujícími se nároky je doba výpočtu vyšší.
Pro zkoumané instance je navíc časová náročnost i u zvýšených požadavků přijatelná -- výsledky jsou téměř okamžité.

Z počátku měření jsem narazil na neodpovídající časovou náročnost způsobenou pravděpodobně neoptimální implementací.
Využití funkcionálních principů bohužel vedlo ke zbytečné alokaci paměti a kopírování objektů (např. aplikací \texttt{map} a \texttt{filter} za sebou vedlo k vytvoření dvou nových kolekcí a jejich dvojitému průchodu) což spolu s velkým množstvím iterací zpomalovalo běh.
I přes částečnou optimalizaci kódu trval výpočet jednotky vteřin i pro malé instance o velikosti $25$ literálů.

Z toho důvodu jsem se rozhodl pro využití optimalizací kompilátoru.
K mému velkému překvapení právě toto vedlo ke snížení časové náročnosti zhruba na jednu padesátinu.

Z uvedených měření mi jako idelání konfigurace vzešly následující hodnoty: \\

\begin{tabular}{ | l | l | }
    \hline
    Parametr & Hodnota \\ \hline \hline
    Počáteční teplota & $300$ \\
    Iterace vnitřní smyčky & $400$ \\
    Rychlost ochlazování & $0.9$ \\ \hline
\end{tabular}

\bibliographystyle{plain}
\bibliography{references}

\end{document}
