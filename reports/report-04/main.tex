\documentclass[czech]{article}

\usepackage[utf8]{inputenc}
\usepackage[IL2]{fontenc}
\usepackage{a4wide}
\usepackage[czech]{babel}
\usepackage{amsmath}
\usepackage{amsfonts}
\usepackage{amstext}
\usepackage{hyperref}
\usepackage{graphicx}

\title{BI-PAA - Problém batohu, úloha 4}
\author{Josef Doležal}

% Illustration
\newcommand{\image}[4][\textwidth]{
    \begin{figure}[h]
        \centering
        \includegraphics[width=#1]{notes/assets/#4}
        \caption{#3}
        \label{fig:#2}
    \end{figure}
}

\begin{document}

\maketitle

\newpage

\section{Úvod}
Problémem batohu se nazývá úloha, ve které je za úkol pro množinu $n$ předmětů a batoh o maximální nosnosti $m$ určit, jak předměty do batohu vložit tak, aby v součtu měly co největší hodnotu a zároveň nebyla překročena maximální nosnost batohu.

Vstupem je tedy seznam $n$ předmětů (dvojic váha-cena) a maximální nosnost batohu $m$.
Výstupem je v součtu nejvyšší možná cena předmětů, jejichž váha dohromady nepřekročí nosnost.

\section{Zadání úlohy}

\begin{itemize}
    \item Zvolte si heuristiku, kterou budete řešit problém vážené splnitelnosti booleovské formule:
    \begin{itemize}
        \item simulované ochlazování,
        \item simulovaná evoluce,
        \item tabu prohledávání.
    \end{itemize}
    \item Tuto heuristiku použijte pro řešení problému batohu. Můžete použít dostupné instance problému, anebo si vygenerujte své.
    \item Hlavním cílem domácí práce je seznámit se s danou heuristikou, zejména se způsobem, jakým se nastavují její parametry a modifikace.
    \item Problém batohu není příliš obtížný, většinou budete mít k dispozici globální maxima (exaktní řešení) z předchozích prací, například z dynamického programování.
\end{itemize}

\section{Rámcový popis řešení}

Pro řešení problému jsem zvolil metodu simulovaného ochlazování.
Simulované ochlazování je iterativní algoritmus, který se snaží v jednotlivých iteracích uniknout z lokálního maxima dočasným přijetím horšího výsledku.

Algoritmus v prvním kroku zvolí náhodné řešení problému (tedy libovolnou konfiguraci, která je řešením) a spočítá jeho hodnotu (součet hodnot vložených položek).
Následně ve smyčce generuje sousední řešení (taková, která se liší pouze v přidání resp. odebrání právě jedné věci).
Pokud je nové řešení lepší, je vždy přijato.
Pokud je horší, je přijato na základě \textit{pravděpodobnosti přijetí}, která je porovnána s náhodným číslem z intervalu $[0, 1]$.
Takto algoritmus pokračuje dokud nedosáhne maximálního počtu kroků.

Algoritmus se skládá ze dvou vnořených smyček.
Vnější reprezentuje aktuální teplotu.
Vnitřní smyčka pro každou teplotu vybírá náhodné sousední řešení.
Počet vnějších smyček je dán ochlazováním aktuální teploty.

Měření jsou prováděna na ukázkových vstupech.

\subsection*{Měnitelné parametry}

Z výše uvedeného popisu algoritmu vyplývá, že měnitelnými parametry algoritmu jsou:

\begin{itemize}
    \item počáteční teplota,
    \item počet hledání sousedního řešení pro aktuální teplotu,
    \item rychlost ochlazování aktuální teploty.
\end{itemize}

Pro každý z nich v práci zkoumám čas běhu a relativní odchylku.

\section{Výsledky měření}

\section{Závěr}

\end{document}
