\documentclass[czech]{article}

\usepackage[utf8]{inputenc}
\usepackage[IL2]{fontenc}
\usepackage{a4wide}
\usepackage[czech]{babel}
\usepackage{amsmath}
\usepackage{amsfonts}
\usepackage{amstext}
\usepackage{hyperref}
\usepackage{graphicx}
\usepackage{float}

\title{BI-PAA - Problém batohu, úloha 4}
\author{Josef Doležal}

% Illustration
\newcommand{\image}[3]{
    \begin{figure}[H]
        \centering
        \includegraphics[width=0.8\textwidth]{assets/#3}
        \caption{#2}
        \label{fig:#1}
    \end{figure}
}

\begin{document}

\maketitle

\newpage

\section{Úvod}
Problémem batohu se nazývá úloha, ve které je za úkol pro množinu $n$ předmětů a batoh o maximální nosnosti $m$ určit, jak předměty do batohu vložit tak, aby v součtu měly co největší hodnotu a zároveň nebyla překročena maximální nosnost batohu.

Vstupem je tedy seznam $n$ předmětů (dvojic váha-cena) a maximální nosnost batohu $m$.
Výstupem je v součtu nejvyšší možná cena předmětů, jejichž váha dohromady nepřekročí nosnost.

\section{Zadání úlohy}

\begin{itemize}
    \item Zvolte si heuristiku, kterou budete řešit problém vážené splnitelnosti booleovské formule:
    \begin{itemize}
        \item simulované ochlazování,
        \item simulovaná evoluce,
        \item tabu prohledávání.
    \end{itemize}
    \item Tuto heuristiku použijte pro řešení problému batohu. Můžete použít dostupné instance problému, anebo si vygenerujte své.
    \item Hlavním cílem domácí práce je seznámit se s danou heuristikou, zejména se způsobem, jakým se nastavují její parametry a modifikace.
    \item Problém batohu není příliš obtížný, většinou budete mít k dispozici globální maxima (exaktní řešení) z předchozích prací, například z dynamického programování.
\end{itemize}

\section{Rámcový popis řešení}

Pro řešení problému jsem zvolil metodu simulovaného ochlazování.
Simulované ochlazování je iterativní algoritmus, který se snaží v jednotlivých iteracích uniknout z lokálního maxima dočasným přijetím horšího výsledku.

Algoritmus v prvním kroku zvolí náhodné řešení problému (tedy libovolnou konfiguraci, která je řešením) a spočítá jeho hodnotu (součet hodnot vložených položek).
Následně ve smyčce generuje sousední řešení (taková, která se liší pouze v přidání resp. odebrání právě jedné věci).
Pokud je nové řešení lepší, je vždy přijato.
Pokud je horší, je přijato na základě \textit{pravděpodobnosti přijetí}, která je porovnána s náhodným číslem z intervalu $[0, 1]$.
Takto algoritmus pokračuje dokud nedosáhne maximálního počtu kroků.

Algoritmus se skládá ze dvou vnořených smyček.
Vnější reprezentuje aktuální teplotu.
Vnitřní smyčka pro každou teplotu vybírá náhodné sousední řešení.
Počet iterací vnější smyčky je dán ochlazováním aktuální teploty.

Měření jsou prováděna na ukázkových vstupech.

\subsection*{Měnitelné parametry}

Z výše uvedeného popisu algoritmu vyplývá, že měnitelnými parametry algoritmu jsou:

\begin{itemize}
    \item počáteční teplota,
    \item počet hledání sousedního řešení pro aktuální teplotu,
    \item rychlost ochlazování aktuální teploty.
\end{itemize}

Pro každý z nich v práci zkoumám čas běhu a relativní odchylku.

\section{Výsledky měření}

Měření znázorňují průměrné hodnoty pro $40$ instancí problému.
Časová složitost byla měřena jako procesorový čas, který byl potřebný přímo k výpočtu.

\subsection{Počáteční teplota}

Počáteční teplotu algoritmu jsem volil podle vzorce $n*k$, kde $n$ je počet položek v batohu a $k$ a parametr měněný v rámci měření.

\subsubsection*{Konfigurace}

\begin{tabular}{ | l | l | }
    \hline
    Parametr & Hodnota \\ \hline \hline
    Počáteční teplota & $[200, 400]$, $k \in 5\dots10$ \\
    Iterace vnitřní smyčky & $40$ \\
    Rychlost ochlazování & $0.9 \cdot \textrm{aktuální teplota}$ \\ \hline
\end{tabular}

\subsubsection*{Rychlost běhu}

\image{dur-initial-temperature}{Časová závislost výpočtu na počáteční teplotě}{dur-initial-temperature.png}

\subsubsection*{Relativní odchylka}

\image{acc-initial-temperature}{Závislost relativní chyby na počáteční teplotě}{acc-initial-temperature.png}

\subsubsection*{Komentář}

Z grafu \ref{fig:dur-initial-temperature} lze vyčíst, že pro malé instance je simulované ochlazování časově zbytečně složité.
Pro instance o alespoň 30 položkách je vidět, že časová složitost nepatrně roste.
Dle očekávání pak mají časovou složitost vyšší ty instance, kde počáteční teplota byla vyšší (dáno vyšším počtem vnějších cyklů).

Graf \ref{fig:acc-initial-temperature} naopak znázorňuje výrazný (i přesto ale linární) růst relativní chyby.
Tento růst je ale vidět u všech počátečních teplot a pravděpodobně je tedy více ovlivněn počtem instancí spíše než teplotou samotnout.

\subsection{Iterace vnitřní smyčky}

\subsubsection*{Konfigurace}

\begin{tabular}{ | l | l | }
    \hline
    Parametr & Hodnota \\ \hline \hline
    Počáteční teplota & $5 \cdot n$ \\
    Iterace vnitřní smyčky & $[100\dots500]$ \\
    Rychlost ochlazování & $0.9 \cdot \textrm{aktuální teplota}$ \\ \hline
\end{tabular}

\subsubsection*{Rychlost běhu}

\image{dur-inner-loop}{Časová závislost výpočtu na průběhu vnitřní smyčky}{dur-inner-loop.png}

\subsubsection*{Relativní odchylka}

\image{acc-inner-loop}{Závislost relativní chyby na průběhu vnitřní smyčky}{acc-inner-loop.png}

\subsubsection*{Komentář}

Obrázek \ref{fig:dur-inner-loop} ukazuje podle očekávání, že časová složitost roste s počtem vnitřních cyklů.
Je ale naopak také patrné, že celkový běh není pro konkrétní velikost vnitřního cyklu nijak výrazně závislý na počtu prvků.

Relativní chyba má dle grafu \ref{fig:acc-inner-loop} rostoucí trend obdobně jako změna počáteční teploty.
Výrazněji je ale vidět, že s roustoucím počtem vnitřních cyklů je průměrná odchylka řádově nižší.
To si lze vysvětlit kvalitnějším vybráním \textit{sousedního} řešení a tedy pravděpodobnějším únikem z lokálního maxima.

\subsection{Rychlost ochlazování}

\subsubsection*{Konfigurace}

\begin{tabular}{ | l | l | }
    \hline
    Parametr & Hodnota \\ \hline \hline
    Počáteční teplota & $5 \cdot n$ \\
    Iterace vnitřní smyčky & $40$ \\
    Rychlost ochlazování & $0.8\dots0.99 \cdot \textrm{aktuální teplota}$ \\ \hline
\end{tabular}

\subsubsection*{Rychlost běhu}

\image{dur-temperature-factor}{Časová závislost výpočtu na rychlosti ochlazování}{dur-temperature-factor.png}

\subsubsection*{Relativní odchylka}

\image{acc-temperature-factor}{Závislost relativní chyby na rychlosti ochlazování}{acc-temperature-factor.png}

\subsubsection*{Komentář}

Z grafů \ref{fig:dur-temperature-factor} a \ref{fig:acc-temperature-factor} lze vyčíst, že přestože rychlým ochlazováním rapidně klesá časová složitost, relativní naopak roste.
Z naměřených výsledků se jako optimální faktor snížení teploty jeví násobek $0.95$, který má velmi nízkou časovou náročnost ale obdobně také nízkou relativní odchylku.

\section{Závěr}

Naměřená data pro časovou náročnost odpovídají předpokladům, že se zvyšujícími se nároky je doba výpočtu vyšší.
Pro zkoumané instance je navíc časová náročnost i u zvýšených požadavků přijatelná -- výsledky jsou okamžité.

Oproti všem očekáváním vyšla ale měření relativní odchylky.
V souladu s měřeními jiných metod řešení (heuristika poměr cena/váha, B\&B, \dots) jsem očekával s rostoucím počtem položek batohu snižující se odchylku.
Ze všech provedených měření ale vyplývá, že chyba roste spolu se zvyšujícím se počtem položek.
Pro ověření správnosti jsem experiment vícekrát opakoval, ale vždy se stejným trendem výsledků.

Z uvedených měření mi jako idelání konfigurace vzešly následující hodnoty: \\


\begin{tabular}{ | l | l | }
    \hline
    Parametr & Hodnota \\ \hline \hline
    Počáteční teplota & $320$ \\
    Iterace vnitřní smyčky & $400$ \\
    Rychlost ochlazování & $0.95$ \\ \hline
\end{tabular}

\end{document}
