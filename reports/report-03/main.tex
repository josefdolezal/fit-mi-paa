\documentclass{article}
\usepackage[utf8]{inputenc}

\title{MI-PAA: Problém batohu, úloha 3}
\author{Josef Doležal}

\usepackage{natbib}
\usepackage[czech]{babel}
\usepackage{a4wide}
\usepackage{graphicx}

\begin{document}

\maketitle

\section{Úvod}
Problémem batohu se nazývá úloha, ve které je za úkol pro množinu $n$ předmětů a batoh o maximální nosnosti $m$ určit, jak předměty do batohu vložit tak, aby v součtu měly co největší hodnotu a zároveň nebyla překročena maximální nosnost batohu.

Vstupem je tedy seznam $n$ předmětů (dvojic váha-cena) a maximální nosnost batohu $m$.
Výstupem je v součtu nejvyšší možná cena předmětů, jejichž váha dohromady nepřekročí nosnost.

\section{Zadání úlohy}

\begin{itemize}
    \item Prozkoumejte citlivost metod řešení problému batohu na parametry instancí generovaných generátorem náhodných instancí. Máte-li podezření na další závislosti, modifikujte zdrojový tvar generátoru.
    \item Na základě zjištění navrhněte a proveďte experimentální vyhodnocení kvality řešení a výpočetní náročnosti.
    \item Zkoumejte zejména následující metody: hrubá síla, metoda větví a hranic, dynamické programování a heuristika (poměr cena/váha).
    Pozorujte zejména závislosti výpočetního času (případně počtu testovaných stavů) a rel. chyby (v případě heuristiky) na: maximální váze věcí, maximální ceně věcí, poměru kapacity batohu k sumární váze a granularitě.
\end{itemize}

\section{Rámcový popis řešení}

Pro určení citlivosti jednotlivých řešení problému je využito generování náhodných vstupů.
U jednotlivých vstupů jsou jsou záměrně měněny parametry generování (např. granularita, počet věcí, maximální cena, \ldots).

Pro jednotlivá řešení jsem vybral parametry takové, u kterých je vysoká pravděpodobnost, že jejich změna bude mít znatelný dopad na celkovou dobu výpočtu (exaktní metody) nebo relativní chybu (heuristika).
Zvolené parametry jsou pro jednotlivé metody následující:

\begin{itemize}
    \item Dynamické programování -- maximální cena,
    \item Metoda větví a hranic:
    \begin{itemize}
        \item maximální váha,
        \item poměr kapacity k maximální váze,
    \end{itemize}

    \item Heuristika:
    \begin{itemize}
        \item maximální cena,
        \item granularita.
    \end{itemize}
\end{itemize}

\section{Popis algoritmu}

\section{Závěr}

\end{document}
